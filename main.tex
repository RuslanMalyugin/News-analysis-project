\documentclass{article}

% Language setting
% Replace `english' with e.g. `spanish' to change the document language
\usepackage[english]{babel}

% Set page size and margins
% Replace `letterpaper' with `a4paper' for UK/EU standard size
\usepackage[letterpaper,top=2cm,bottom=2cm,left=3cm,right=3cm,marginparwidth=1.75cm]{geometry}

% Useful packages
\usepackage{amsmath}
\usepackage{graphicx}
\usepackage[colorlinks=true, allcolors=blue]{hyperref}

%%% Работа с русским языком
\usepackage{cmap}					% поиск в PDF
\usepackage{mathtext} 				% русские буквы в формулах
\usepackage[T2A]{fontenc}			% кодировка
\usepackage[utf8]{inputenc}			% кодировка исходного текста
\usepackage[english,russian]{babel}	% локализация и переносы
\usepackage{indentfirst}
\frenchspacing

\usepackage{hyperref}
\usepackage[usenames,dvipsnames,svgnames,table,rgb]{xcolor}
\hypersetup{				% Гиперссылки
    unicode=true,           % русские буквы в раздела PDF
    pdftitle={Заголовок},   % Заголовок
    pdfauthor={Автор},      % Автор
    pdfsubject={Тема},      % Тема
    pdfcreator={Создатель}, % Создатель
    pdfproducer={Производитель}, % Производитель
    pdfkeywords={keyword1} {key2} {key3}, % Ключевые слова
    colorlinks=true,       	% false: ссылки в рамках; true: цветные ссылки
    linkcolor=red,          % внутренние ссылки
    citecolor=black,        % на библиографию
    filecolor=magenta,      % на файлы
    urlcolor=cyan           % на URL
}

\documentclass{article}

% Language setting
% Replace `english' with e.g. `spanish' to change the document language
\usepackage[russian]{babel}

% Set page size and margins
% Replace `letterpaper' with`a4paper' for UK/EU standard size
\usepackage[letterpaper,top=2cm,bottom=2cm,left=3cm,right=3cm,marginparwidth=1.75cm]{geometry}

% Useful packages
\usepackage{amsmath}
\usepackage{graphicx}
\usepackage{listings}
\usepackage{amssymb}
\usepackage[
backend=biber,
style=alphabetic,
sorting=ynt
]{biblatex}
\addbibresource{sample.bib}


\title{Анализ влияния новостей на стоимость акций}
\author{Малюгин Руслан}

\begin{document}
\maketitle

\renewcommand{\abstractname}{Аннотация}

\begin{abstract}
В данной работе приведен обзор инструментов для анализа и сбора новостей и информации о стоимости акций. Рассматривается несколько решений для обработки и дальнейшего анализа текста. Производится анализ и проверка гипотез о влиянии новостей на изменение стоимости акций. Так же реализован код для мониторинга новостей и получения прогноза в реальном времени.
\end{abstract}

\section{Обзор инструментов}

\subsection{Сбор новостей}
В качестве источника новостей был выбран сайт CNBC. Этот сайт один из самых популярных, соответственно здесь большой охват аудитории и, возможно,новости отсюда влияют на людей при покупке и продаже ценных бумаг.

У сайта нет официального API, но есть библиотека для работы - cnbc. Она имеет лимиты на запросы в месяц, но позволяет получать новости о конкретных компаниях.

В дальнейшем работа ведется с этой библиотекой.

\subsection{Обработка новостей}

Для обработки новостей были попробованы 2 метода: построение эмбеддингов новости и после обучения модели на них, и использование предобученной модели Roberta. Это модель, разработанная компанией Twitter для анализа тональности. Для работы первого метода необходим датасет для обучения, в связи с чем был написан бот для парсинга новостей и составления датасета. Для хорошей работы потребуется время для сборы данных, так что на данный момент анализ ведется с помощью второго метода.

\section{Гипотезы о влиянии новостей}

Для проверки были выдвинуты следующие гипотезы: при негативном окрасе новости цена акции падает, и наоборот, при позитивном возрастает, и то, что изменение цены в случае негативного или позитивного окраса новости сильнее, чем в случае нейтрального изменения.

Для эксперимента были выбраны компании Google, Meta, Apple и Tesla, так как они являются наиболее популярными и про них больше новостей.

\subsection{Гипотеза Negative vs Positive}

В качестве $H_0$ будем предполагать, что в случае оценки новости как Negative, через сутки цена упадет, а в случае Positive - вырастет. Обозначим $\mu _{t}$ - цена в момент выхода новости, $\mu _{t+1}$ - цена через день. Таким образом имеем следующие бернуллиевские величины:

$\xi _{i} =\begin{cases}
0,\ Prediction\ =\ Negative,\ \mu _{t} < \mu _{t+1}{} & \\
1,\ Prediction\ =\ Negative,\ \mu _{t}  >\mu _{t+1}{} & \\
1,\ Prediction\ =\ Positive,\ \mu _{t} < \mu _{t+1}{} & \\
0,\ Prediction\ =\ Positive,\ \mu _{t}  >\mu _{t+1}{} & 
\end{cases}$

В качестве гипотезы рассматриваем $H_{0} :\theta \  >\ 0.5\ vs\ H_{1} :\theta < 0.5$

Для проверки гипотезы воспользуемся биномиальным тестом. 
Он состоит в том, что если $\xi _{i} \sim Bern( \theta )$, то $\sum _{i=1}^{n} \xi _{i} \sim Bin( n,\theta )$. После этого мы можем рассмотреть 0.95 квантиль распределения и сделать вывод.

В экспериментах были получены следующие величины:

$\hat{\theta } =0.63,\ n=27$

После проверки гипотез было получено p-value на уровне 0.94, что говорит нам о том, что мы более уверены в первой гипотезе, нежели в ее отвержении(важно помнить, что мы не можем утверждать о принятии первой гипотезы, так как выборка была рассмотрена небольшая).

\subsection{Гипотеза Neutral vs NonNeutral}

В данном случае будем замерять изменении цены при нейтральном окрасе новости и изменении цены при положительном или негативном окрасе. Обозначим за $\xi _{i} =|\mu _{t} -\mu _{t+1}|$ - изменение цены, $\overline{\xi _{NN}}$ - среднее изменение при положительном или негативном окрасе (NN = NonNeutral), и $\overline{\xi _{N}}$ - среднее изменение при нейтральном окрасе.
Будем рассматривать гипотезы $H_{0} :E( \xi _{NN} -\xi _{N}) \ =\ 0\ \ vs\ H_{1} :E( \xi _{NN} -\xi _{N}) \  >\ 0$

Для теста воспользуемся предположением о нормальности выборок(это неслишком строгое требование и в целом все хорошо работает и в случае распределений, близких к нормальному) и используем t-test.

T-test состоит в следующем. Пусть у нас есть 2 случайные величины $X_{1}$ и $X_{2}$ с матожиданиями $M_{1}$, $M_{2}$, и размерами выборок $n_{1}$ и $n_{2}$ соответственно. Будем проверять гипотезу $H_{0} : M_{1} - M_{2}  \ =\ 0\ $. Введем статистку $t=\frac{\overline{X}_{1} -\overline{X}_{2}}{\sqrt{\frac{s_{1}}{n_{1}} +\frac{s_{2}}{n_{2}}}}$, где $s_{1}$, $s_{2}$ - выборочные дисперсии, $\overline{X}_{1}$, $\overline{X}_{2}$ - выборочные матожидания. При справедливости нулевой гипотезы эта статистика имеет распределение Стьюдента. После этого, аналогично первого случая сравниваем с квантилями и выносим решение.

В ходе нашего эксперимента были получены $\overline{\xi _{NN}}$ = 1.37 и $\overline{\xi _{N}}$ = 1.17. После проверки было получено p-value на уровне 0.67, что говорит о том, что вероятно, новости все-таки как-то влияют на изменении цены(важно помнить, что мы не можем утверждать о принятии первой гипотезы, так как выборка была рассмотрена небольшая).

\section{Тестирование модели с новостными признаками}
Для тестирования был выбран датасет с ценами на акции Apple, собранный за несколько лет. Его большой минус - это интервалы длиной день. Но в данной работе нам важно не построить модель для точного предсказания, а проверка того, что новости положительно влияют на прогнозирование.


Для экспериментов была выбрана базовая модель CatBoost. В качестве признаков будем иметь: временные(день, месяц, час), прошлые(допустим, это будет 10 прошлых измерений), новостные признаки(2 признака - вероятность Negative и Positive, выданные моделью, в случае отсутствия новостей = ставим нули). Таргет будет бинарный - цена вверх или вниз.

После обучения модели были получены следующие показатели качества по метрике accuracy: 0.682 и 0.696. Как показываю результаты эксперимента, даже в такой простой модели, добавление новостных признаков, помогло нам улучшить качество, хоть и не сильно, но при более серьезном подходе к построению модели, эмбеддингов, сбору данных можно получить более хороший результат.

\section{Заключение}

В ходе проделанной работы были опробованы многие библиотеки для работы с новостями и финансовыми данными. Реализован метод для сбора датасетов с новостями. Собраны и найдены датасеты с новостями и финансовыми данными. Описаны возможные подходы для решения задачи. Проверены гипотезы о значимости новостей на курс акций. После этого были построены модели и зафиксировано улучшение качества модели при добавлении новостных признаков.

В дальнейшем планируется собрать хороший датасет и испробовать другие методы работы с текстами, а так же более мощный модели для предсказания.



\bibliographystyle{alpha}
\begin{thebibliography}{3}
\bibitem{1}
\cite{https://link.springer.com/article/10.1007/s42001-019-00035-x}
\bibitem{2}
\cite{https://arxiv.org/abs/0809.2792}
\bibitem{3}
\cite{https://econpapers.repec.org/article/gamjecomi/v_3a8_3ay_3a2020_3ai_3a4_3ap_3a107-_3ad_3a457822.htm}
\bibitem{4}
\cite{https://www.mdpi.com/2227-7099/8/4/107/htm}
\bibitem{5}
\cite{https://www.diva-portal.org/smash/get/diva2:1637576/FULLTEXT01.pdf}
\bibitem{6}
\cite{http://statistica.ru/theory/t-kriterii/}
\bibitem{7}
\cite{https://github.com/topics/cnbc-api}
\bibitem{8}
\cite{https://huggingface.co/cardiffnlp/twitter-roberta-base-sentiment}
\bibitem{9}
\cite{https://www.kaggle.com/datasets/lorilaz/apple-news-headline-sentiment-and-stock-info}

\end{thebibliography}

\bibliography{}

\end{document}